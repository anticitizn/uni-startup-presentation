\documentclass[handout]{beamer}
%Information to be included in the title page:
\title{Artificial Intelligence oder Bullshit?}
\subtitle{Erkennen von Betrug und echten Geschäftsmöglichkeiten}
\author{Daniel Neykov}

\usepackage{xcolor}
%\usetheme{Copenhagen}

\setbeamertemplate{background}
{
    \includegraphics[width=\paperwidth,height=\paperheight]{background.png}
}
\setbeamertemplate{frametitle}{
  \vspace{0.9cm} % Increase or decrease the vertical space as needed
  \hspace{-0.5cm}
  \insertframetitle
}

\begin{document}

\frame{\titlepage}

\begin{frame}
\frametitle{KI Überblick}

\begin{itemize}
    \item<1-> KI-Markt wächst um $>20\%$ jährlich
    \item<2-> Jährlich viele neue Anwendungsbereiche für KI entdeckt
    \item<3-> ... aber auch viele KI-Startups erweisen sich als Bullshit!
    \item<4-> Bla, bla, bla
\end{itemize}
\end{frame}

\begin{frame}
    \frametitle{AI oder BS?}
    \begin{itemize}
        \item<1-> Hindsight is always 20/20 \dots
        \item<2-> wie schwer kann es denn sein, festzustellen, ob ein Startup einen echten Wert hat oder nicht?
    \end{itemize}
\end{frame}

% Company: Vicarious
\begin{frame}
    Produkt: \\
    \begin{itemize}
        \item<1-> Software zur Interpretation von Bildern, Ton und Text auf nahezu menschlichem Niveau
        \item<2-> Fähigkeit, ein Gespräch mit einem Menschen zu führen und eine Vielzahl komplexer Aufgaben zu lösen
        \item<3-> $>\$50$ Millionen Finanzierung
    \end{itemize} ~\\
    \only<4>{
        \frametitle{\includegraphics[width=0.3\textwidth]{vicarious.png}}
        \begin{itemize}
            \item[\textcolor{red}{$\blacktriangleright$}] Bullshit!
            \item[\textcolor{red}{$\blacktriangleright$}] Unternehmen existiert noch, aber keine nennenswerten Produkte und die finanzielle Zukunft ist fraglich
        \end{itemize}
    }
\end{frame}

\begin{frame}
    \frametitle{\includegraphics[width=0.3\textwidth]{vicarious.png}}
    Woran ist es gescheitert? \\
    \begin{itemize}
        \item<1-> Die Schlüsselpersonen sind in erster Linie Venture Capitalists und keine Techniker
        \item<2-> Sehr anspruchsvolles und umfassendes Ziel
        \item<3-> Kein klarer und präziser Plan, wie dieses Ziel erreicht werden soll
    \end{itemize} ~\\
\end{frame}

% Company: DeepMind
\begin{frame}
    Produkt: \\
    \begin{itemize}
        \item<1-> Entwicklung künstlicher allgemeiner Intelligenz
        \item<2-> \dots indem sie Software schreiben, die Videospiele spielt
        \item<3-> $>\$50$ Millionen Finanzierung
    \end{itemize} ~\\
    \only<4>{
        \frametitle{\includegraphics[width=0.3\textwidth]{deepmind.png}}
        \begin{itemize}
            \item[\textcolor{green}{$\blacktriangleright$}] Success!
            \item[\textcolor{green}{$\blacktriangleright$}] Wurde von Google für \$600 Millionen aufgekauft, erfand Deep Q Learning, produzierte AlphaGo, AlphaFold, etc.
        \end{itemize}
    }
\end{frame}

\begin{frame}
    \frametitle{\includegraphics[width=0.3\textwidth]{deepmind.png}}
    Warum ist es erfolgreich? \\
    \begin{itemize}
        \item<1-> Gegründet von führenden Forschern
        \item<2-> Ambitionierter langfristiger Fokus
        \item<3-> \dots aber präzise und erreichbare kurzfristige Ziele
    \end{itemize} ~\\
\end{frame}

% Company: DeepL
\begin{frame}
    Produkt: \\
    \begin{itemize}
        \item<1-> Zweisprachiges Online-Wörterbuch
        \item<2-> Bietet begrenzte AI-gestützte Unterstützung
        \item<3-> Selbstfinanziert
    \end{itemize} ~\\
    \only<4>{
        \frametitle{\includegraphics[width=0.3\textwidth]{deepl.png}}
        \begin{itemize}
            \item[\textcolor{green}{$\blacktriangleright$}] Success!
            \item[\textcolor{green}{$\blacktriangleright$}] Unabhängig erfolgreich geblieben, übertrifft alle bestehenden Tech-Giganten in der Qualität der Übersetzungen
        \end{itemize}
    }
\end{frame}

\begin{frame}
    \frametitle{\includegraphics[width=0.3\textwidth]{deepl.png}}
    Warum ist es erfolgreich? \\
    \begin{itemize}
        \item<1-> Gegründet von erfahrenen Ingenieuren
        \item<2-> Einsatz der vorhandenen traditionellen Dienste (Wörterbuch) zur Verbesserung der KI-Übersetzungsdienste
        \item<3-> Sehr gut definiertes und fokussiertes Ziel
    \end{itemize} ~\\
\end{frame}

% Company: Mayfield Robotics
\begin{frame}
    Produkt: \\
    \begin{itemize}
        \item<1-> Kleiner, mobiler Roboter, der Mimiken darstellen kann
        \item<2-> Sprach- und Gesichtserkennung, autonomes Fahren
        \item<3-> Unterstützt von Bosch
    \end{itemize} ~\\
    \only<4>{
        \frametitle{\includegraphics[width=0.3\textwidth]{mayfield.png}}
        \begin{itemize}
            \item[\textcolor{red}{$\blacktriangleright$}] Bullshit!
            \item[\textcolor{red}{$\blacktriangleright$}] Trotz des großen Hypes am Anfang und der ordentlichen Vorbestellungen wurde Mayfield Robotics nach drei Jahren geschlossen
        \end{itemize}
    }
\end{frame}

\begin{frame}
    \frametitle{\includegraphics[width=0.3\textwidth]{mayfield.png}}
    Woran hat es gescheitert? \\
    \begin{itemize}
        \item<1-> Keine wirkliche Innovation gegenüber Konkurrenzprodukten wie Alexa, Google Assistant usw.
        \item<2-> Geringe Gewinnspannen
        \item<3-> \dots dennoch ein hoher Preis von 799\$
    \end{itemize} ~\\
\end{frame}

% Company: OpenAI
\begin{frame}
    Produkt: \\
    \begin{itemize}
        \item<1-> Förderung und Entwicklung von "freundlicher KI" und der dafür erforderlichen Infrastruktur
        \item<2-> Förderung der Zusammenarbeit zwischen vielen Organisationen im Bereich KI
        \item<3-> $>$ \$1 Milliarden Finanzierung
    \end{itemize} ~\\
    \only<4>{
        \frametitle{\includegraphics[width=0.3\textwidth]{openai.png}}
        \begin{itemize}
            \item[\textcolor{green}{$\blacktriangleright$}] Success!
            \item[\textcolor{green}{$\blacktriangleright$}] Weitere Finanzmittel in Milliardenhöhe, alle freigegebenen kommerziellen Produkte erlangen sofort eine enorme Zugkraft usw.
        \end{itemize}
    }
\end{frame}

\begin{frame}
    \frametitle{\includegraphics[width=0.3\textwidth]{openai.png}}
    Warum ist es erfolgreich? \\
    \begin{itemize}
        \item<1-> Riesiges Startkapital
        \item<2-> Trotz umfassender langfristiger Ziele liegt der Schwerpunkt auf praktischen und erreichbaren kurzfristigen Ergebnissen
        \item<3-> Kernteam aus weltweit Top-Forschern
    \end{itemize} ~\\
\end{frame}

% Company: Cambridge Analytica
\begin{frame}
    \vspace{-0.5cm}
    Produkt: \\
    \begin{itemize}
        \item<1-> Datengesteuerte Marketing- und Social-Media-Strategien
        \item<2-> Ausrichtung auf öffentlichkeitswirksame politische Kampagnen
        \item<3-> ~\$15 Millionen Finanzierung
    \end{itemize} ~\\
    \only<4>{
        \frametitle{\includegraphics[width=0.2\textwidth]{cambridge-analytica.png}}
        \begin{itemize}
            \item[\textcolor{red}{$\blacktriangleright$}] Bullshit!
            \item[\textcolor{red}{$\blacktriangleright$}] Obwohl das Unternehmen zunächst ein hohes Wachstum verzeichnete, ging es 2018 pleite
        \end{itemize}
    }
\end{frame}

\begin{frame}
    \frametitle{\includegraphics[width=0.2\textwidth]{cambridge-analytica.png}}
    Woran hat es gescheitert? \\
    \begin{itemize}
        \item<1-> Die viel gepriesenen Fähigkeiten ihres KI-Algorithmus erwiesen sich als völlig überbewertet und wurden durch einfache statistische Methoden besiegt
        \item<2-> Skandale um Datenmissbrauch führten zu hohen Geldstrafen, Ermittlungen und dem Verlust von Kunden
    \end{itemize} ~\\
\end{frame}

% Skript-teil
\begin{frame}
    \frametitle{Wie unterscheide ich?}
    \begin{itemize}
        \item<1-> Ist der Hintergrund des Kernteams passend?
        \item<2-> Ist der Zeitplan für die Umsetzung realistisch?
        \item<3-> Bietet es etwas neues im Vergleich zur Konkurrenz?
        \item<4-> \dots ist das überhaupt legal?
    \end{itemize} ~\\
\end{frame}

\begin{frame}
    \frametitle{Bonuspunkte für Startups}
    \begin{itemize}
        \item<1-> Offen für externe Audits und Verifizierung
        \item<2-> Besitzen Urheberrechte und Patente
        \item<3-> Guter Hintergrund der Kapitalgeber
    \end{itemize} ~\\
\end{frame}

\begin{frame}
    \frametitle{Q\&A}
    \pause
    Danke für die Aufmerksamkeit!
\end{frame}


    
\end{document}
